\documentclass[12pt]{article}%

%============================================================================
% Assignment Information

\def\assignmentName { Project 2 - Website Analysis                      }
\def\className      { CSE 321: Internet and Web Programming             }
\def\studentName    { Noah Schoonover \\ Myles Willis \\ Adam Schmidt   }
\def\studentEmail   {  }
\def\dueDate        { August 26, 2021                                   }
\def\semesterDate   { Fall 2021                                         }

%============================================================================
% Packages

\usepackage{tabularx, makecell}
\usepackage{setspace}
\usepackage[shortlabels]{enumitem}
\renewcommand{\thesubsection}{\thesection.\alph{subsection}}
\usepackage[a4paper, top=2.5cm, bottom=2.5cm, left=1.5cm, right=1.5cm]{geometry}
\usepackage{amsmath, amssymb}
\usepackage{float, graphicx}
\usepackage[bottom]{footmisc}

%============================================================================
% User Defined Commands

\newcommand{\itemAt}[1]{\setcounter{enumi}{#1 - 1}\item}
\newcommand{\ntab}{\hspace*{1cm}}

%============================================================================
% Headers and Document Setup

\doublespacing

%============================================================================
\begin{document}
%============================================================================

%---------------------------------------------------------------------
% Title

\begin{singlespace}
\title{ \assignmentName }
\author{ \studentName \\ {\small \studentEmail} \\ {\it \className}}
\date{\dueDate (\semesterDate)}
\maketitle
\end{singlespace}

%---------------------------------------------------------------------
% Document


\begin{itemize}
    \item Theme
    \item Group Name
    \item Group Members
\end{itemize}

\begin{enumerate}

    \item Background
    \item Website Analysis
        \begin{enumerate}
            \item 
            \item 
            \item \hspace{1cm} \\
            \begin{tabular}{|l|l|l|l|l|l|l|}
                \hline
                 & Proposed System & System 1 & System 2 & System 3 & System 4 & System 5 \\
                 \hline
                Function 1 & \\
                Function 2 & \\
                Function 3 & \\
                Function 4 & \\
                 \hline
            \end{tabular}
        \end{enumerate}

    
\end{enumerate}

%============================================================================
\end{document}
%============================================================================